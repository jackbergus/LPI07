\documentclass[]{scrartcl}
\usepackage[utf8]{inputenc}
\usepackage[T1]{fontenc}
\usepackage{amsmath}
\usepackage{xcolor}
\title{Lab Lesson 02}
\date{November 7, 2017}
\author{Giacomo Bergami}
\begin{document}
\maketitle
\section*{Exercises}

\begin{enumerate}
\item Scrivere un programma che legga sempre un numero per ogni linea.
Interrompere l'immissione dell'utente al quinto valore pari.
Visualizzare la somma di tutti i numeri ottenuti.
\item Scrivere un programma che legga sempre un numero per ogni linea.
Sommare ciascun numero ai precedenti, solo se è dispari.
Visualizzare la somma di tutti i numeri ottenuti appena raggiunge, o supera, 100.
\item Scrivere un programma per calcolare il valore massimo e minimo di un insieme
di N numeri inseriti da tastiera. Ciascun numero verrà fornito in una riga distinta.
Il programma deve leggere il valore di N, ed in
seguito deve leggere una sequenza di N numeri. A questo punto il programma
deve stampare il massimo ed il minimo tra i numeri inseriti.
\item Creare un programma che genera l'n-esimo numero della sequenza di Fibonacci. Ricordati che:
\begin{itemize}
	\item $Fib(0) = 1$
	\item $Fib(1) = 1$
	\item $\forall n\geq 2. Fib(n) = Fib(n-1)+Fib(n-2)$
\end{itemize}
\item Scrivere un programma per poter analizzare una sequenza di numeri separati da spazi in una stessa linea.
Si vogliono calcolare e stampare su schermo diversi risultati:
\begin{itemize}
	\item quanti sono i numeri positivi, nulli e negativi
	\item  quanti sono i numeri pari e dispari
	\item  se la sequenza dei numeri inseriti è crescente, decrescente oppure non monotona. Distinguere i casi di sequenza vuota e quando questa è costituita da un solo elemento.
\end{itemize}

\item Dato un intero letto da terminale, stampare la lista di tutti i suoi divisori (interi)
\item Scrivere un programma che controlli se una stringa fornita in input è palindroma
\item Creare un programme che, date due stringhe arbitrarie ``data'' e ``pattern'', controlla quante volte ``pattern'' è contenuto dentro a ``data''.


\end{enumerate}
\end{document}
