\documentclass[]{scrartcl}
\usepackage[utf8]{inputenc}
\usepackage[T1]{fontenc}
\usepackage{amsmath}
\usepackage{xcolor}
\title{Lezione di Laboratorio 03}
\date{21 Novembre, 2017}
\author{Giacomo Bergami}
\begin{document}
\maketitle
\section*{Exercises}

\begin{enumerate}
\item  Leggere da terminale una stringa di double arbitraria separata da uno spazio, e valutare il valore massimo.
\item Creare un programma che legge una matrice da terminale.
Chiedere prima all'utente il numero di righe e colonne, leggere la matrice per righe e controllare che il numero di celle inserite corrisponda con il numero di colonne.
Stampare la matrice letta da terminale

Utilizzare \texttt{System.out.printf("\%nd", i);} per stampare interi composti da massimo n cifre
\item Modifica l'esercizio precedente per sommare due matrici
\item  Modifica l'esercizio precedente per moltiplicare due matrici
\item Crea un quadrato magico usando il metodo Siamese per matrici di dimensioni dispari N, riempiendolo con numeri da 1 a $N^2$.
 Riempi la matrice a partire dalla cella centrale della prima riga con il numero 1,
 poi per riempire le scatole muoversi diagonalmente in alto e a destra ($ \nearrow$), un passo alla volta. Quando una mossa
 uscirebbe dal quadrato, ritorna all'ultima riga o alla prima colonna. Riempi ogni cella con un
 numero crescente. Se si incontra una cella piena, ci si sposta verticalmente in basso di una casella ($\downarrow$), quindi continua
 come prima.

 Alla fine, fai un po' di debug, e controlla se la matrice ottenuta è un quadrato magico, ogni riga e ogni diagonale
 devono avere lo stesso valore di somma.


\end{enumerate}
\end{document}
