\documentclass[]{scrartcl}
\usepackage[utf8]{inputenc}
\usepackage[T1]{fontenc}
\usepackage{amsmath}
\usepackage{xcolor}
\usepackage{braket}
\usepackage{booktabs}
\usepackage{hyperref}
\usepackage{verbatim}
\title{Lab Lesson XY}
\author{Giacomo Bergami}
\begin{document}
\maketitle
\section{Ricorsione ed Extends}
Si vuole creare un metodo ricorsivo che, dato un array di caratteri, inverta l'ordine degli elementi. Similarmente, data una stringa, si vuole invertire una stringa. Osserva che banalmente potrei creare due metodi, che però sostanzialmente eseguono lo stesso processo (o algoritmo). È possibile definire un unico algoritmo, implementato sia per array di caratteri sia per stringhe? \textbf{Osserva:} utilizzare le classi e l'ereditarietà per creare una classe di oggetti invertibili, che possono essere sia stringhe sia array di caratteri.

\section{Numeri di Stirling di seconda specie}
Scrivere una funzione \texttt{stirling} che, dati due interi $n$ e $k$, stampi i numeri di Stirling di seconda specie come da definizione della seguente funzione:
\[S(n,k)=\frac{1}{k!}\sum_{j=0}^k(-1)^{k-j}\binom{k}{j}j^n\]

In \url{https://it.wikipedia.org/wiki/Numeri_di_Stirling#Seconda_specie} trovate una tabella di esempio che fornisce i valori attesi per $0\leq n,k\leq 9$.

\section{Rubrica}
Una rubrica telefonica è composta da più voci; ciascuna voce è formata da un nome e da un numero di telefono. Si vuole creare una classe driver che consenta all'utente di effettuare le seguenti azioni su richiesta dell'utente:
\begin{itemize}
\item Inserire una voce inesistente (nome/numero di telefono);
\item Trovare il numero di telefono associato ad un determinato nome;
\item Stampare l'intera rubrica
\item Uscire dal driver
\end{itemize}

\section*{Mini Database (Generalizzazione dell'esercizio della Esercitazione 9)}
Si vuole creare un programma che legga da file due tipi di file: un file di \texttt{dati} ed un file di \texttt{interrogazioni}. 

Il file di \texttt{dati} fornisce una rappresentazione tabulare,
dove ogni colonna è separata da una virgola (tranne l'ultima colonna), ed ogni riga è delimitata dalla usuale andata a capo.
La prima riga contiene sempre un'\textit{intestazione}, ovvero definisce il nome di ciascuna colonna. Tutte le righe rimanenti contengono dei valori numerici, sempre separati da virgole. Ad esempio, il file in Figura \ref{fig:a} 
dovrà essere caricato in memoria come la tabella in Figura \ref{fig:b}.

\begin{figure}[!b]
	\centering
	\begin{minipage}{.3\textwidth}
		\centering
		\begin{verbatim}
		A,B,C
		1,2,3
		1,2,4
		4,5,6
		4,5,7
		5,5,7
		7,8,7
		7,11,23
		\end{verbatim}
		\caption{File \texttt{dati}}
		\label{fig:a}
	\end{minipage}%
	\begin{minipage}{0.4\textwidth}
		\centering
		\begin{tabular}{ccc}
			\toprule
			A &B&C\\
			\midrule
			1&2&3\\
			1&2&4\\
			4&5&6\\
			4&5&7\\
			5&5&7\\
			7&8&7\\
			7&11&23\\
			\bottomrule
		\end{tabular}
	\caption{Rappresentazione tabulare }
		\label{fig:b}
	\end{minipage}\begin{minipage}{0.4\textwidth}
	\centering
	\begin{tabular}{cc}
		\toprule
		A &B\\
		\midrule
		1&2\\
		1&2\\
		4&5\\
		4&5\\
		5&5\\
		7&8\\
		7&11\\
		\bottomrule
	\end{tabular}
	\caption{\texttt{project A,B}}
	\label{fig:h}
\end{minipage}
\end{figure}

Il file di interrogazione fornisce invece un modo per manipolare i \texttt{dati} caricati in memoria, e per generare nuove tabelle. Il database è inizializzato con la tabella vuota senza intestazione. Ogni comando di tale linguaggio di interrogazione è scritto in una linea differente. La lista completa dei comandi è fornita qui sotto.
\begin{itemize}
	\item \texttt{LOAD filename}
	
	Carica una tabella \texttt{dati} da un file salvato nella posizione \texttt{filename}. Se il file non esiste, rimane in memoria la tabella precedentemente caricata. Una tabella senza intestazione (e quindi, senza righe) è una stringa di lunghezza zero.
	\item \texttt{PRINT}
	
	Stampa con \texttt{System.out.println} la rappresentazione della tabella in memoria come in Figura \ref{fig:a}. 
	\item \texttt{DUMP filename}
	
	Salva la stessa stringa che viene stampata dal metodo precedente in un file denominato \texttt{filename}.
	
	\item \texttt{COUNT}
	
	Stampa a terminale una nuova linea contenente il numero di righe tella tabella in memoria.
	
	\item \texttt{PROJECT $a_1$,...,$a_n$}
	
	Aggiorna la tabella caricata in memoria selezionando quali colonne devono essere restituite. Tali colonne vengono selezionate tramite il loro nome. Osserva, una stesso nome di colonna non deve essere ripetuto; se così fosse, non si applica la proiezione e la tabella corrente rimane quella precedentemente caricata (Figura 3). Se esiste un $a_i$ non presente nell'intestazione, $a_i$ viene semplicemente non restituito nella tabella finale: se non esiste nessuna colonna, si può restituire una tabella vuota.
	
\begin{figure}[!t]
	\centering
	\begin{minipage}{.3\textwidth}
		\centering
\begin{tabular}{ccc}
	\toprule
	A &B&C\\
	\midrule
	7&8&7\\
	7&11&23\\
	\bottomrule
\end{tabular}
		\caption{\texttt{FiLtER B > 5}}
		\label{fig:c}
	\end{minipage}%
	\begin{minipage}{0.5\textwidth}
		\centering
		\begin{tabular}{cc}
			\toprule
			A &K\\
			\midrule
			1&4\\
			4&10\\
			5&5\\
			7&19\\
			\bottomrule
		\end{tabular}
		\caption{\texttt{GROUP B by A wiTH + as K}}
		\label{fig:d}
	\end{minipage}
\end{figure}
	\item \texttt{FILTER $a_i$ op $j$}
	
	La tabella correntemente caricata in memoria mantiene solamente quelle righe le cui celle corrispondenti alla colonna di nome $a_i$ hanno un valore $k$, tale per cui la valutazione di ``$k$ \texttt{op} $j$'' è vera. In particolare, si vogliono interpretare solo gli operatori \texttt{op} di confronto $<$, $>$, $<=$, $>=$, $=$ e $!=$. Se la colonna $a_i$ non è presente nella tabella, si restituisce la tabella vuota con l'intestazione precedente.
	
	\item \texttt{GROUP $a_i$ BY $a_j$ WITH op AS $a_k$}
	
	Se $a_i$ e $a_j$ esistono nella tabella corrente e se $a_j\neq a_k$, restituisce una nuova tabella $\tilde{t}$ con intestazione $\{a_j,a_k\}$; altrimenti, si restituisce la tabella vuota senza intestazione. 
	
	Per ottenere $\tilde{t}$, nella tabella già caricata si ``raggruppano'' tutte le righe che hanno lo stesso valore numerico $k$ per $a_j$ e, su tutti i valori di $a_j$, si effettua l'operazione \texttt{op}, ottenendo $v$. 	\texttt{op} può essere un'operazione di somma su array o di prodotto su array. Per ogni valore $k$, si restituisce la riga $\{k,v\}$.
\end{itemize}
Ciascuna della parte dei comandi scritti in maiuscolo deve essere valuata in modo case unsensitive.

Si vuole inoltre definire una classe \texttt{Client} che interpreti i comandi scritti da terminale come un \textit{driver}, ed una \texttt{Interpreter} che esegua questi comandi quando scritti su file. Si pensi ad attuare una politica di riuso del codice, che consenta di non replicare il codice due volte. Si utilizzi quindi una classe madre/padre \texttt{MiniQueryParser}, che venga estesa da entrambe le classi\dots
\bigskip

Come autovalutazione, questa esercitazione vi fornisce tre cartelle ed una classe:
\begin{itemize}
\item La cartella \texttt{table} contenente tre semplici tabelle
\item La cartella \texttt{query} contenente otto esempi di interrogazioni.
\item La cartella \texttt{expected} contenente il rusultato atteso stampato da terminale dal vostro codice.
\item La classe \texttt{TestCorrectTable}, che effettua l'autovalutazione.
\end{itemize}
Si fornisce inoltre un Makefile per snellire la procedura di compilazione (\texttt{make}), pulitura dei file class inutili (\texttt{make clean}) e dell'esecuzione del codice (\texttt{make run}).

\end{document}
