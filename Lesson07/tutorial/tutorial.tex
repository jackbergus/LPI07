\documentclass[]{scrartcl}

%opening
\title{Come risolvere un problema ricorsivo}
\author{Giacomo Bergami}
\usepackage{braket}
\newcommand{\prog}{\ensuremath{\mathcal{P}_{f,g}}}
\usepackage[utf8]{inputenc}
\usepackage[italian]{babel}
\usepackage{hyperref}
\usepackage{amsmath,amsfonts}

\begin{document}

\maketitle


\section{}
Una buona definizione ricorsiva di un problema $\prog$ si poggia su due definizioni chiave, ovvero quella di ``caso base'' e ``caso induttivo''. Questo approccio viene utilizzato anche il logica ed in matematica discreta per fornire dimostrazioni\footnote{\url{https://it.wikipedia.org/wiki/Principio_d\%27induzione}} ``strutturali'' (se ben applicato, consente di ottenere dimostrazioni corrette, e quindi anche programmi corretti\footnote{\url{http://www-sop.inria.fr/marelle/Laurent.Thery/formal/curry.html}}).
\begin{itemize}
	\item Si definisce \textbf{caso base} un dato $d$ per il quale il programma riesce a fornire immediatamente una risposta, senza ulteriori chiamate ricorsive. Matematicamente:
	\[\prog(d)=f(d)\]
	\item Si definisce \textbf{caso induttivo} un dato $d$ che può essere scomposto in due componenti, $\Braket{d_1,d_2}$, dove $d_1$ può essere valuato immediatamente mentre $d_2$ richiede una valutazione di $\prog(d_2)$. La valutazione di $\prog(d)\equiv \prog(\Braket{d_1,d_2})$ è data da una valutazione parziale di $d_1$, che deve essere combinata con quella restituita da $\prog(d_2)$, sul quale si applica la ricorsione. Più matematicamente:
	\[\prog(\Braket{d_1,d_2})=g(f(d_1),\prog(d_2))\]
\end{itemize}
 Come si può osservare, l'obiettivo dell'applicazione dell'induzione è quello di applicare la definizione ricorsiva di $\prog$ solamente su una sotto-componente di $d$, ovvero $d_2$, in modo che si raggiunga sempre il caso base. In questo modo la ``correttezza'' garantisce anche che non si verifichino mai Stack-Overflow.
\medskip

\textbf{Esempio.\quad }\textit{Ogni numero naturale $\mathbb{N}$ può essere descritto induttivamente tramite un caso base, zero $0$, ed un qualsiasi numero successivo allo zero $(n)\mathbf{+1}$ Data questa definizione triviale di numero intero, potremmo pensare alla definizione del seguente algoritmo (inefficiente ma logicamente corretto): per la somma di due numeri $n$ ed $m$}
\begin{itemize}
	\item Se $n=0$, allora $n+m=m$
	\item Se $n=\tilde{d_2}+d_1$ con $d_1=1$, allora valuta $\tilde{n}+m$ e poi incrementa questo risultato di uno.
\end{itemize}
\textit{In questo caso avremo che il programma somma $\prog(n,m)$ avrà come funzione $f$ la costante $1$ ($f(x)=1$) e $g$ la funzione somma così definita $g(f(1),\prog(d_2,m))=f(1)+\prog(d_2,m)=\prog(d_2,m)\mathbf{+1}$. Quindi avremo:}
\[\prog(n,m)=\begin{cases}
	m & n = 0\\
	\prog(d_2,m)\mathbf{+1} & n=d_2\mathbf{+1}\\
\end{cases}\]
\textit{Osservate che la funzione restituisce un numero somma utilizzando la definizione induttiva di numero naturale.}

\subsection{Esercizio}
A questo punto, proviamo a risolvere il problema \textbf{scrivere un programma con metodo ricorsivo che conti il numero di cifre dispari in un numero}. Proviamo quindi a decomporre questo problema in un problema ricorsivo che utilizza il principio d'induzione:
\begin{itemize}
\item Se il numero è negativo, lo converto in un numero positivo, così da non ripetere la procedura induttiva di analisi delle cifre sia per il caso negativo, sia per il caso positivo.
\item Se il numero è positivo, procedo come utilizzando l'induzione:
\begin{enumerate}
	\item Se il numero è compresto tra $0$ e $9$ compresi, allora il problema si riduce a valutare se il numero è pari o dispari.
	\[\prog(d)=f(d)\qquad f(d)=\texttt{(}d \texttt{\;\% 2 == 0 ? 0 : 1)}\]
	\item Altrimenti, mi posso chiedere se la cifra meno significativa sia dispari, quindi $d_1 = d\texttt{\;\%10}$. Se quest'ultima cifra è dispari, allora restituisco uno e zero altrimenti. Poiché $(d\texttt{\;\%10})\texttt{\;\%2}=d\texttt{\;\%2}$, allora abbiamo che $f(d_1)=\prog(d_1)$.
	
	Da definizione precedente, abbiamo che $d_2 = d\texttt{\;/10}$, in quanto dobbiamo analizzare la parte rimanente del numero, escludendo la cifra meno significativa.
	
	Alla fine, sommiamo il numero di cifre dispari di $d_2$ con il valore di $f(d_1)$, e quindi $g$ è la funzione somma: $g(x,y)=x+y$. 
\end{enumerate}
\end{itemize}
Otterremo quindi la seguente definizione matematica:
\[\prog(d)=\begin{cases}
\prog(-d) & d < 0\\
\texttt{(}d \texttt{\;\% 2 == 0 ? 0 : 1)} & 0\leq d\leq 9\\
f(d\texttt{\;\%10})+\prog(d\texttt{\;/10}) & d\geq 10\\
\end{cases}\]
o equivalentemente:
\[\prog(d)=\begin{cases}
\prog(-d) & d < 0\\
\texttt{(}d \texttt{\;\% 2 == 0 ? 0 : 1)} & 0\leq d\leq 9\\
\prog(d\texttt{\;\%10})+\prog(d\texttt{\;/10}) & d\geq 10\\
\end{cases}\]

Si veda il sorgente di \texttt{CheckIfOdd.java} per una sua implementazione.

\section{Tail Recursion}
Una definizione ricorsiva come quella precedente, benché sia di immediata definizione, non è efficiente. Nella maggior parte dei casi, la soluzione ottimale è quella di non calcolare il risultato assieme alla chiamata ricorsiva, ma di passare a tale chiamata ricorsiva tale soluzione parziale $r$. Otteniamo quindi la seguente definizione:

\[\prog'(d,r)=g(f(d),r)\]
\[\prog'(\Braket{d_1,d_2},r)=\prog'(d_2,g(f(d_1),r))\]

Tuttavia, questo richiede che si conosca quale sia il valore $\overline{r}$ di $r$ da fornire alla prima applicazione della funzione $\prog'$. Quindi, otterremo un algoritmo ricorsivo che utilizza il principio di induzione nel seguente modo:

\[\prog(d)=\prog'(d,\overline{r})\]

\subsection{Esercizio}
Proviamo ora a riscrivere l'esercizio precedente utilizzando un approccio tail recursive. Nel nostro caso avremo che  $\overline{r}=0$, perché alla prima esecuzione non avremo la possibilità di analizzare nessuna cifra. 

Conseguentemente, avremo:
\[\prog(d)=\prog'(d,0)\]
\[\prog'(d,r)=\begin{cases}
\prog'(-d,r) & d < 0\\
r+\texttt{(}d \texttt{\;\% 2 == 0 ? 0 : 1)} & 0\leq d\leq 9\\
\prog'(d\texttt{\;/10},r+\texttt{(}d \texttt{\;\% 2 == 0 ? 0 : 1)}) & d\geq 10\\
\end{cases}\]

Si veda il sorgente di \texttt{CheckIfOddTailRecursive.java} per una sua implementazione.
\medskip

\textbf{Osserva.}\textit{È possibile applicare l'induzione anche in senso inverso, ovvero analizzando la prima cifra del numero intero invece dell'ultimo. Quest'ultimo richiede l'utilizzo della funzione logaritmo della classe \texttt{Math}. Viene lasciata per esercizio.}

\end{document}
